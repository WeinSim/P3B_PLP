\documentclass{article}

\usepackage[utf8]{inputenc}
\usepackage[german]{babel}
\usepackage[
    a4paper, top=2.5cm, bottom=2.5cm, left=2.5cm, right=2.5cm, marginparwidth=1.75cm
]{geometry}
\usepackage{amsmath}
\usepackage{amsfonts}
\usepackage{enumitem}
\usepackage[
    colorlinks=true, 
    citecolor=black,
    filecolor=black,
    linkcolor=black,
    urlcolor=blue
]{hyperref}
\usepackage{graphicx}
\usepackage{amssymb}
\usepackage{float}
\usepackage{pdfpages}
\usepackage{multirow}
\usepackage[
    tocskip=0.1\baselineskip, skip=0.7\baselineskip, parfill
]{parskip}
\usepackage{listings}
\usepackage{fancyhdr}
\usepackage{xcolor}

% Kopf- und Fußzeile
\pagestyle{fancy}
\fancyhf{}
%Kopfzeile mittig mit Kaptilname
\fancyhead[C]{\nouppercase{\leftmark}}
%Fußzeile links bzw. innen
\fancyfoot[L]{\versuchsname}
%Fußzeile mittig (Seitennummer)
\fancyfoot[R]{\thepage}
\renewcommand{\footrulewidth}{0.35pt}

% Hilfs-Commands für Gleichungen
\newcommand{\widespace}{\enspace}
\newcommand{\wideeq}{\widespace = \widespace}
\newcommand{\wideneq}{\widespace \neq \widespace}
\newcommand{\wideapprox}{\widespace \approx \widespace}
\newcommand{\wideleq}{\widespace \leq \widespace}
\newcommand{\widegeq}{\widespace \geq \widespace}
\newcommand{\widele}{\widespace \le \widespace}
\newcommand{\widege}{\widespace \ge \widespace}
\newcommand{\wideiff}{\widespace \iff \widespace}
\newcommand{\wideimplies}{\widespace \implies \widespace}
\newcommand{\pd}[2]{
    \frac{\partial #1}{\partial #2}
}
\newcommand{\result}[2]{
    #1 \, \text{#2}
}

% Markieren von verwendetem Code
\definecolor{codebg}{RGB}{230, 240, 255}
\newcommand{\coderef}[1]{
    \text{
        \enspace
        \footnotesize
        \colorbox{codebg}{\texttt{#1()}}
    }
}

% Formatieren von Code im Anhang
\definecolor{codegreen}{rgb}{0,0.6,0}
\definecolor{codegray}{rgb}{0.5,0.5,0.5}
\definecolor{codepurple}{rgb}{0.58,0,0.82}
\definecolor{backcolour}{rgb}{0.95,0.95,0.92}
\lstdefinestyle{mystyle}{
    backgroundcolor=\color{backcolour},   
    commentstyle=\color{codegreen},
    keywordstyle=\color{magenta},
    numberstyle=\tiny\color{codegray},
    stringstyle=\color{codepurple},
    basicstyle=\ttfamily\footnotesize,
    breakatwhitespace=false,         
    breaklines=true,                 
    captionpos=b,                    
    keepspaces=true,                 
    numbers=left,                    
    numbersep=5pt,                  
    showspaces=false,                
    showstringspaces=false,
    showtabs=false,                  
    tabsize=4
}
\lstset{style=mystyle}

% Formatierung von Absätzen
\renewcommand{\baselinestretch}{1.2}

\allowdisplaybreaks

% Allgemeine Infos
\newcommand{\versuchsname}{
    PLP - Plasmaphysik
}
\newcommand{\githuburl}{
    \url{https://github.com/WeinSim/P3B}
}

% Titel und Autor
\title{\versuchsname}
\author{Simon Weinzierl, Yannic Werner}

\begin{document}

\maketitle

\begin{center}
    Physikalisches Fortgeschrittenenpraktikum P3B
    nach der Studienordnung für Studienbeginn bis WS 2022/23
\end{center}

\vspace*{6cm}

\begin{center}
    \footnotesize
    Alle Teile dieses Dokuments (Vorbereitung, Protokoll, Auswertung) wurden
    von beiden Teilnehmern in gleichen Teilen und ohne fremde Hilfe bearbeitet.
    Sofern fremde Quellen verwendet wurden, sind diese angegeben.

    Der \LaTeX-Code ist auf GitHub unter \githuburl verfügbar.
    
    © Alle Rechte vorbehalten.
\end{center}

 % LMU-Siegel
\AddToShipoutPicture*{
    \put(315,0){
        \parbox[b][5cm]{5cm}{
            \includegraphics[width=10cm]{Abbildungen/Siegel_LMU.pdf}
        }
    }
}

\newpage

% Inhaltsverzeichnis
\tableofcontents

\newpage

% Literatur
\bibliographystyle{alpha}
\bibliography{literatur}

\newpage

\section{Vobereitung}

\subsection{Physikalischer Hintergrund}


\newpage

\section{Veruschsablaufplan}

\subsection{Benötigte Materialien}
    \begin{enumerate}[label=\arabic*.]
        \item Eins
        \item Zwei
    \end{enumerate}

\newpage

\subsection{Teilversuch 1: Bragg-Reflexion von Röntgenstrahlung des Molybdän an einem NaCl-Eiskristall}
\begin{enumerate}[label = (\Roman*)]
    \item Ziel: ...
    
    \item Versuchsmethode: ...
    
    \item Versuchsskizze:
    
        \begin{figure}[H]
        \centering
        \includegraphics[width=0.7\linewidth]{Bild}
        \caption{Versuchsskizze Teilversuch 1}
        \end{figure}

    \item Planung der Durchführung
        \begin{itemize}
            \item eins
            \item zwei
        \end{itemize}

    \item Vorüberlegungen zur Durchführung \& Auswertung
        \begin{itemize}
            \item eins
            \item zwei
        \end{itemize}
    
\end{enumerate}

\newpage

\subsection{Teilversuch 2: Energiespektrum einer Röntgenröhre in Abhängigkeit der Spannung}
\begin{enumerate}[label = (\Roman*)]
    \item Ziel: 
    
    \item Versuchsmethode: 
    
    \item Versuchsskizze:
    
        \begin{figure}[H]
        \centering
        \includegraphics[width=0.7\linewidth]{Bild}
        \caption{Versuchsskizze Teilversuch 2}
        \end{figure}

    \item Planung der Durchführung
        \begin{itemize}
           \item eins
           \item zwei
        \end{itemize}

    \item Vorüberlegungen zur Durchführung \& Auswertung
        \begin{itemize}
            \item eins
            \item zwei
        \end{itemize}
        
\end{enumerate}


\newpage

\section{Versuchsprotokoll}

Auf den folgenden Seiten befindet sich das eingescannte Versuchsprotokoll.
Alle Daten wurden selbst gemessen. Sofern fremde Hilfe benutzt wurde,
wurde sie klar gekennzeichnet.

Messunsicherheiten wurden angegeben und folgend in der Auswertung verwendet.
Alle weiteren Rechnungen und Analysen finden in der Versuchsasuwertung statt.

\includepdf[pages={...}, pagecommand={\thispagestyle{scrheadings}}, frame=true]{[Name.pdf]]}

\newpage

\section{Auswertung}

    \subsection{Teilversuch 1: Bragg-Reflexion von Röntgenstrahlung des Molybdän an einem NaCl-Eiskristall}

    \newpage

\section{Anmerkung: Graphische Auswertung und Fehlerfortpflanzung mit Python-Code}

Alle Berechnungen inkl. Fehlerbestimmung wurden mit einem selbstgeschriebenen
Python-Skript durchgeführt, um uns die Arbeit zu erleichtern und Fehler zu
vermeiden. Alle Ergebnisse, die auf diese Weise zustande gekommen sind,
sind entsprechend mit einem \colorbox{codebg}{blauen Hintergrund} gekennzeichnet;
s. folgendes Beispiel:
\[
    F \wideeq ma \wideeq \result{20}{kg} \cdot 9,81 \, \frac{\text m}{\text s^2}
    \wideeq \result{(19,62 \pm 0,5)}{N} \coderef{tv1}
\]
Dies soll bedeuten, dass die Berechnung des Wertes und der Unsicherheit von der
Python-Funktion namens \verb|tv1| durchgeführt wird.
Die Unsicherheit wird mithilfe der Gauß'schen Fehlerfortpflanzung berechnet.
Außerdem wird das Python-Package \texttt{matplotlib} zum Erstellen
von Graphen verwendet.

Der verwendete Code ist sowohl auf GitHub verfügbar (\githuburl) als auch auf den
folgenden Seiten zu finden und kann mit dem Befehl \texttt{python Main.py}
ausgeführt werden. Für eine genauere Beschreibung des Codes siehe die README-Datei
auf GitHub sowie die Kommentare im Code.
(Manche Sonderzeichen im Code (ä, ö, ü, $\Delta$, etc.) werden von \LaTeX nicht
richtig erkannt, deswegen kann der Code auf den nachfolgenden Seiten an einigen
Stellen unvollständig erscheinen. Auf GitHub wird aber alles richtig angezeigt.)

\newpage


\verb|Main.py|:
\lstinputlisting[language=Python]{Code/Main.py}
\newpage

\verb|Expressions.py|:
\lstinputlisting[language=Python]{Code/Expressions.py}
\newpage

Output:
\lstinputlisting{Code/Output.txt}

\end{document}